\section{Inverse Reinforcement Learning}\label{IRL}

To understand better the aim of Inverse Reinforcement Learning we describe firstly what Reinforcement Learning is.

In a way, Reinforcement Learning is the science of making optimal decisions using experiences. Breaking it down, the process of Reinforcement Learning involves these simple steps: the agent takes from the environment the current state; it computes the best action; the agent performs it in the environment. The goal of RL is to find an optimal behavior strategy for the agent to obtain optimal rewards.
The main idea behind the RL approach is the \textsl{Markov decision process}. This process is defined by a tuple $( S, A, R, p, \gamma)$:

\begin{equation}\label{eq:Markov}
     p(s', r | s, a) = Pr[ S_{t+1} = s', R_{t+1} = r | S_{t} = s, A_{t} = a ] 
\end{equation}

\begin{equation}\label{eq:discounted}
    G_{t} = R_{t+1} + \gamma R_{t+2} + {\gamma}^2 R_{t+3} + \dots
\end{equation}
     


 where $S_{t}, S_{t+1} \in S$ are the states, $A_{t+1} \in A$ is an action, $R_{t}, R_{t+1} \in R$ are the rewards, p is the process dynamism and $G_{t}$ is the discounted reward. The discounted rewards indicates to the agent the best trajectory to achieve the goal. \\
 The Markov decision process \ref{eq:Markov}, in other words, defines the transition probability in a new state, taking some rewards from the current state with the execution of an action. 
Another important component is the discount factor $\gamma$. It defines the penalty to uncertainty of future rewards, so larger $\gamma$ increases the sensibility to the future rewards.
This coefficient value is in $(0,1)$, but typically is close to $1$.

In the RL is defined a \textsl{policy network} that implements the agent. This policy network $\pi$ define an actions probability distribution given the states:

\begin{equation}
    \pi( A_{t} = a | S_{t} = s), \forall A_{t} \in A, S_{t} \in S
\end{equation}

This creates a \textsl{trajectory}, that is an ordered sequence of states, actions and rewards: $S_{0}, A_{0}, R_{1}, S_{1}, A_{1}, R_{2}, S_{2}, A_{2}, R_{3} \dots$.
The agent, therefore, tends to maximize the "expected" reward following the policy network $\pi$. To do that, the policy is modelled with a parameterized function respect to $\theta$. This function is  $\pi_{\theta}(a|s)$ and represents the probability of make the action \textit{a} given the state \textit{s}. \\ 
We can now define the \textsl{Reinforcement Learning Objective}: maximize the "expected" reward following a parameterized policy:

\begin{equation}
    J(\theta) = E_{\pi_{\theta}} [r(\tau)]
\end{equation}

where $r(\tau)$ define the \textsl{total reward} for a given trajectory $\tau$. \\
A standard method to find the $\theta$ that optimize J is the \textsl{Gradient Ascent} method. Here $\theta$ is updated like that:

\begin{equation}
    \theta_{t+1} = \theta_{t} + \alpha\nabla_{\theta} J(\theta)
\end{equation}

Using gradient ascent, we can move $\theta$ toward the direction suggested by the gradient $\nabla_{\theta}J(\theta)$ to find the best $\theta$ for $\pi_{\theta}$ that produces the highest return. Computing the gradient $\nabla_{\theta}J(\theta)$ is tricky because it depends on both the action selection and the stationary distribution of states following the target selection behavior \cite{gradient}. The only problem is that since the environment is unknown, it is difficult to estimate the effect on the state distribution by a policy update.
For this reason, it is used the \textsl{Policy Gradient Theorem} which simplify the gradient computation  $\nabla_{\theta}J(\theta)$:

\begin{equation}\label{eq:nablae}
    \nabla E_{\pi_{\theta}}[r(\tau)] = E_{\pi_{\theta}}[r(\tau)\nabla log(\pi_{\theta}(\tau))]
\end{equation}

where:

\begin{equation}\label{eq:pitau}
    \pi_{\theta}(\tau) = P(s_{0}) + \prod_{t=1}^{T} \pi_{\theta}(a_{t}|s_{t})p(s_{t+1}, r_{t+1} | s_{t}, a_{t})
\end{equation}

In equation \ref{eq:pitau}, \textit{P} represents the \textsl{ergodic distribution} which starts from some state $s_{0}$. \\
Every step, an action \textit{a} is taken according to $\pi_{\theta}$ and, based on that, the environment dynamic \textit{p} decides which state transition has to be performed. Then this is multiplied by the trajectory length T. Now consider the logarithm of equation \ref{eq:pitau}, we have that $log\pi_{\theta}(\tau)$ is:

\begin{equation*}
    \log P(s_{0}) + \sum_{t=1}^{T} log\pi_{\theta}(a_{t}|s_{t}) + \sum_{t=1}^{T} \log p(s_{t+1},r_{t+1}|s_{t}, a_{t}) 
\end{equation*}

from which we can see that:

\begin{equation}\label{eq:nablalog}
    \nabla \log\pi_{\theta}(\tau) = \sum_{t=1}^{T}\nabla\log\pi_{\theta}(a_{t}|s_{t})
\end{equation}

Now putting together the equations\ \ref{eq:nablae} and\ \ref{eq:nablalog}, we obtain the final policy gradient algorithm:

\begin{equation}\label{eq:policygradient}
    \nabla E_{\pi_{\theta}}[r(\tau)] = E_{\pi_{\theta}}[r(\tau)(\sum_{t=1}^{T} \nabla log\pi_{\tau})]    
\end{equation}

Replacing $r(\tau)$ from the equation\ \ref{eq:policygradient} with the discounted rewards $G_{t}$\ (\ \ref{eq:discounted}), we obtain the classic policy gradient algorithm called \textsl{Renforce}:

\begin{equation}
    \nabla E_{\pi_{\theta}}[r(\tau)] = E_{\pi_{\theta}}[\sum_{t=1}^{T} G_{t} \nabla log\pi_{\tau}] 
\end{equation}

In IRL, the policy does not receive the reward directly from the environment. Instead,  we assume that there is a human in the loop who has an intention for the agent’s task, and communicates this intention to the agent using one feedback channel, that is the Preferences\ \cite{NIPS2018_8025}. The preferences are pairs of short trajectory segments of agent's behaviour which the human compares, preferring those that are closer to the intended goal.
The preferences are collected during the experiment while the policy is training. 
In this case, are introduced two new actors: an \textit{annotator} and a \textit{reward model}. The former gives preference feedback; the latter estimates a reward function from the annotator’s preferences which will give rewards to the policy during its training.
